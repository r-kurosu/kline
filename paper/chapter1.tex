\chapter{はじめに}

複数の港で自動車を積み,複数の港で自動車を降ろす自動車運搬船について考える.
一般的に自動車運搬船は,自動車を船の一定間隔で区切られたホールドと呼ばれるスペースにどの自動車を何台割り当てるかを決定する席割作業と呼ばれる工程を経て,その後席割作業で割り当てられた自動車に対して向きや場所を考慮して一台ずつ船内の領域に配置するシミュレーションと呼ばれる作業を行う.
現状,自動車を輸送する会社はこの作業を人手で行なっていることが多く,席割作業に3時間,シミュレーション作業に4時間かかることから,これらの工程を自動化することが業務効率化に役立つと考えている.

本研究では,2つの工程のうちシミュレーション作業に対して,数理最適化の技術によってコンピュータで短時間かつ効率の良いシミュレーション結果を出力することを目標とする.

本研究で扱うシミュレーション作業の概要について述べる.
席割作業の結果から,各階層の各ホールドにどの種類の車を何台詰込むかという情報が与えられる.
この情報をシミュレーション作業の入力値として扱う.

ホールド内に詰込む種類の車は一種類とは限らないので,ホールド内のどの位置に車を詰込むかといったことを決定する.


本稿では,,,