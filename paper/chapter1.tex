\chapter{はじめに}

複数の港で自動車を積み, 複数の港で自動車を降ろす自動車運搬船について考える. 
一般的に自動車運搬船は, 自動車を船の一定間隔で区切られたホールドと呼ばれるスペースにどの自動車を何台割り当てるかを決定する席割と呼ばれる工程を経て, その後席割で割り当てられた自動車に対して向きや場所を考慮して一台ずつ船内の領域に配置するシミュレーションと呼ばれる作業を行う. 
貨物である自動車は積載位置まで自走するため, 積み降ろしのタイミングでは搬入・搬出経路が通行可能になっている必要がある. 
また, 航海中の船体復元性及び荷役安全性を考慮する必要がある. 
現状, 自動車を輸送する会社はこの作業を人手で行なっていることが多く, 席割作業に3時間, シミュレーション作業に4時間かかると言われている. \cite{mitsui}. 
プランナーごとの経験値や技量によって積み付け計画の品質に個人差が生じ, 急な状況の変化による積み付け計画の変更に業務負荷が生じるなどの問題が存在している. 
本研究では, 2つの工程のうちシミュレーションに対して, 数理最適化の技術を用いて効率の良い結果を出力することを目標とする. 

本研究で扱うシミュレーションの概要について述べる. 
席割の結果から, 各階層の各ホールドにどの種類の車を何台詰め込むかという情報が与えられる. 
この情報をもとにシミュレーションを行う. 
シミュレーションは, 車の横幅と前後の長さを辺とする長方形に近似することで長方形詰込み問題として定式化できる. 
本研究では, 二段階に分けた構築法を提案する. 
一段階目では, 各ホールド内に詰め込む車を, 積み地や揚げ地の情報をもとにグループ分けし, グループごとにデッキ内での大まかな配置場所を決める. 
二段階目では, グループ内における車両一台ずつの詳細な配置場所を決定する. 

本稿では, 第2章で問題に対する詳細な設定や, 本研究で扱う自動車運輸航海における専門用語について定義する. 
第3章では, 問題の定式化と定式化に必要な変数や定数の定義を行う. 
第4章と第5章では, 提案手法の説明と解の考察を行う. 
