\chapter{はじめに}

複数の港で自動車を積み, 複数の港で自動車を降ろす自動車運搬船について考える. 
一般的に自動車運搬船は, 自動車を船の一定間隔で区切られたホールドと呼ばれるスペースにどの自動車を何台割り当てるかを決定する席割と呼ばれる工程を経て, その後席割で割り当てられた自動車に対して向きや場所を考慮して一台ずつ船内の領域に配置するシミュレーションと呼ばれる作業を行う. 
現状, 自動車を輸送する会社はこの作業を人手で行なっていることが多く, 席割作業に3時間, シミュレーション作業に4時間かかることから, これらの工程を自動化することが業務効率化に役立つと考えている. 

本研究では, 2つの工程のうちシミュレーションに対して, 数理最適化の技術によってコンピュータで短時間かつ効率の良いシミュレーション結果を出力することを目標とする. 

本研究で扱うシミュレーションの概要について述べる. 
席割の結果から, 各階層の各ホールドにどの種類の車を何台詰め込むかという情報が与えられる. 
この情報をもとにシミュレーションを行う. 

シミュレーションは, 車を長方形に近似することで長方形詰込み問題として定式化できる. 
本研究では, 二段階に分けた構築法を提案する. 
一段階目では, 各ホールド内に詰め込む車を, 積み地や揚げ地の情報をもとにグループ分けし, グループごとに大まかな配置場所を決める. 
二段階では, グループ内における車両一台一台の詳細な配置場所を決定する. 

本稿では, 第2章で問題に対する詳細な設定や, 本研究で扱う自動車運輸航海における専門用語について定義する. 
第3章では, 問題の定式化と定式化に必要な変数や定数の定義を行う. 
第4章では, 提案手法の詳細な説明を行う. 
