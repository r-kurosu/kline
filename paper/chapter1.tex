\chapter{はじめに}
これは卒論や修論のテンプレートです.
卒論や修論を作る前に,書き物に関する基本的な注意点を書いたページ
\footnote{\url{http://www.co.cm.is.nagoya-u.ac.jp/~yagiura/writing/writing.html}}
に書いたことをよく読んでください.
このページには加筆,訂正することがときどきあるので,資料を作るたびに見直してください.

発表のたびに配布資料を書く(あるいは前回の資料に加筆修正する)ことで,
論文のための文章を書くことや\LaTeX を使うことに慣れるとともに,
卒論・修論につながる文章を蓄積していってください.

このテンプレートはあくまでも参考ですので,
フォントサイズや行間等のスタイルを自由に変更してかまいません.

「はじめに」の節ではこの資料で何を書くのかを手短に説明してください.
どのような問題を対象にしてどのような手法を提案し,どのような結果が得られたのかを書くなどです.
また,研究背景についても関連研究の文献を挙げつつ述べましょう.
文献リストの書き方の例を挙げておきます.
\cite{GareyJohnson79}は本,
\cite{MHI13,YIG04}は論文誌の論文,
\cite{IYI05}は国際会議の予稿集の論文,
\cite{JohnsonMcGeoch97}は論文を集めた本やハンドブックに掲載された論文の例です.

必要な情報を
順序よく(つまり後ろを見ないと分からないことが出て来たりしないように),
明確に(曖昧さなく厳密に),
コンパクトに(冗長な表現を無くして手短に)書くよう心がけましょう.


構成は研究テーマや書きたいことによって様々ですが,
例えばある問題に対するアルゴリズムを提案して,
その効果を計算実験によって検証するようなテーマであれば,はじめにの節ののち,
問題の説明,提案手法の説明,計算実験の紹介,まとめという構成がしばしば用いられます.
以下の節ではそのような説明を書く際によく利用する書式や書くべきことの例をいくつか示しておきます.