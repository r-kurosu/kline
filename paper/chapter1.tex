\chapter{はじめに}

複数の港で自動車を積み, 複数の港で自動車を降ろす自動車運搬船について考える. 
一般的に自動車運搬船は, 自動車を船の一定間隔で区切られたホールドと呼ばれるスペースにどの自動車を何台割り当てるかを決定する席割と呼ばれる工程を経て, 
その後席割作業で割り当てられた自動車に対して向きや場所を考慮して一台ずつ船内の領域に配置するシミュレーションと呼ばれる作業を行う. 
貨物である自動車は積載位置まで自走するため, 積み降ろしのタイミングでの搬入・搬出経路を確保する必要がある. 
また, 航海中の船体バランス及び荷役安全性を考慮する必要がある. 
現状, 自動車を輸送する会社はこの作業を人手で行なっていることが多く, 席割作業に3時間, シミュレーション作業に4時間かかると言われている\cite{mitsui}. 
プランナーごとの経験値や技量によって積み付け計画の品質に個人差が生じ, 急な状況の変化による積み付け計画の変更に業務負荷が生じるなどの問題が存在している. 
名古屋大学の竹田が積み付け計画自動化の第一歩として, 席割作業の自動化に関する研究を進めている\cite{takeda}. 
本研究では積み付け計画自動化の次の段階としてシミュレーション作業自動化を目的とし, 短時間で実用的なシミュレーション結果を出力することを目指す.  

本研究で扱うシミュレーション作業の概要について述べる. 
入力として, 各階層の各ホールドにどの種類の車を何台詰め込むかという情報が与えられる. 
この情報をもとにシミュレーション作業を行う. 
シミュレーション作業は, 車の幅と長さを各辺とする長方形に近似することで長方形詰込み問題として定式化できるが, 貨物である自動車は積載位置まで自走するということを考える必要がある. 
そのため, 積み降ろしのタイミンングでの搬入搬出経路の確保と, 駐車時に要する局所的スペースの確保が課題となる. 

本研究では, 二段階に分けた構築法と局所探索を提案する. 
一段階目では, 各ホールド内に詰め込む車を, 積み地や揚げ地の情報をもとにグループ分けし, グループごとにデッキ内での大まかな配置場所を決める. 
二段階目では, グループ内における車両一台ずつの詳細な配置場所を決定する. 
以上の二段階による構築法が終わった後, 簡単な局所探索を行い解を改善する. 

本稿では, 第2章で問題に対する詳細な設定や, 本研究で扱う自動車運輸航海における専門用語について定義する. 
第3章では, 問題の定式化と定式化に必要な変数や定数の定義を行う. 
第4章と第5章では, 提案手法の説明と解の考察を行う. 

