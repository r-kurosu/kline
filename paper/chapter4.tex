\chapter{提案手法}\label{method}
本研究では2段階に分けた手法を提案する.
1段階目では,同一セグメント内で,積み地,揚げ地が同じ車を一つのグループとし,デッキ内のグループの配置場所を決定する.
2段階目では,各グループ内で一台ずつの詳細な配置場所を決定する.


\section{第一段階}
同一セグメントで,積み地揚げ地が同じ車を一つのグループとする.
この際,グループ内の車の総面積を計算しておく.

\subsection{シーケンスペアによる解表現}
シーケンスペアとは長方形詰込み問題において使われる手法である.
二つの順列を元に,左右上下の相対位置関係を決定することができ,一意の配置図を生成することができる.
【具体的には以下の通り】

\subsection{soft-rectangleパッキング問題}
長方形は高さと幅が決められた形を指す.
soft-rectangleは,面積一定のもと高さと幅の長さを変えることのできる長方形である.
パッキング問題の応用として,soft-rectangleパッキング問題がある.

\subsection{第一段階に対する解法}
グループを積み地の昇順に並べた順列∂+と,揚げ地の降順に並べた順列∂ーを用意する.
∂+において,積み地が同じグループが存在する場合,それらのグループ間で揚げ地の降順にする.
∂-において,揚げ地が同じグループが存在する場合,それらのグループ間で積み地の昇順にする.

二つの順列で順序が同じグループは,上下方向の制約を追加する.
二つの順列で順序が異なるグループは,左右方向の制約を追加する.

以上にsoft-rectangleパッキング問題の制約を加え,整数計画ソルバー(Gurobi Optimizer)を用いて計算を行った.
与えられた問題例の全てで,1秒以内に最適解を求めることができた.


\section{第二段階}
第一段階で定められた領域内で,車一台ずつの詳細な配置場所を決定する.

\subsection{bottom-left法}
bottom-left法(BL法)とは長方形詰込み問題に対するアルゴリズムの一つで,配置可能な領域の内,最も下に,同じ高さの候補があれば,最も左に配置する手法である.
面積の大きい順にパッキングするといったように,順番を工夫することで有用な解が得られるとされている.

本研究では,解の構築にbottom-left法を応用した手法を用いた.
デッキの入口から見て下の部分では,bottom-left法をそのまま利用した.
デッキの入口から見て上の部分では,配置可能な場所のうち出来るだけ上,同じ高さなら左といったtop-left法を採用した.



\subsection{レクトリニア図形}
レクトリニア図形とは,縦の線分と横の線分で囲まれた図形である.
レクトリニア図形は長方形の輪集合としても表される.

本研究ではパッキング時に,駐車に必要なスペースを加えたレクトリニア図形と見なし,パッキング後は元の大きさの長方形とすることで一台一台の駐車位置を確保した.
