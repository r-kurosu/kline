\chapter{提案手法}\label{method}
本章では, 第3章で定式化した二段階の問題についてそれぞれの問題に対する提案手法を説明する. 

\section{第一段階}
求解には整数計画ソルバー (Gurobi Optimizer ver. 9.5)を用いた. 
第二段階では, この時の出力結果を利用する.  

\section{第二段階}
\subsection*{bottom-left法}
bottom-left法 (BL法)とは長方形詰込み問題に対するアルゴリズムの一つで, 配置可能な領域の内, 最も下に, 同じ高さの候補があれば, 最も左に配置する手法である\cite{nfp2}. 
面積の大きい順にパッキングするといったように, 詰め込む順番を工夫することで有用な解が得られるとされている. \\


本研究では, 車の駐車向きを考慮し, 搬出時ランプに向かって前向きに発進できるようにBL法を修正し, 詰め込みを行った. 
各グループ内で, 全長の大きいもの, その中でも幅の大きいものを先に, そして同じ車種が連続するような順番で詰め込みを行った. 
グループ間の詰め込みの順番としては, 積み地の早い順, 積み地が同じ場合, 揚げ地の遅い順に詰め込みを行った. 

グループ内の全ての車が詰め終わるか, 詰め込み途中で, グループ内のどの車も詰め込むことができない場合, 計算を終了し, 次のグループ内の車の詰め込みを開始した. 
