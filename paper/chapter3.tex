\chapter{定式化}\label{formulation}

この章では本研究における記号の定義や問題の定式化について説明する.\\

\section{記号の定義}

\begin{table}[htb]
    \begin{center}
    \label{table31}
    \begin{tabular}{cp{35em}} \hline
    記号 & \hspace{2.0em}記号の説明 \\ \hline

    $x_i$ &
    \begin{tabular}{l}
    \hspace{1.4em}長方形$i$の左下のx座標
    \end{tabular} \\ \hline

    $y_i$ &
    \begin{tabular}{l}
    \hspace{1.4em}長方形$i$の左下のy座標
    \end{tabular} \\ \hline    
    
    $w_i$ &
    \begin{tabular}{l}
    \hspace{1.4em}長方形$i$の幅
    \end{tabular} \\ \hline

    $w_i^l$ &
    \begin{tabular}{l}
    \hspace{1.4em}長方形$i$の幅の下界
    \end{tabular} \\ \hline

    $w_i^u$ &
    \begin{tabular}{l}
    \hspace{1.4em}長方形$i$の幅の上界
    \end{tabular} \\ \hline

    $h_i$ &
    \begin{tabular}{l}
    \hspace{1.4em}長方形$i$の高さ
    \end{tabular} \\ \hline

    $h_i^l$ &
    \begin{tabular}{l}
    \hspace{1.4em}長方形$i$の高さの下界
    \end{tabular} \\ \hline

    $h_i^u$ &
    \begin{tabular}{l}
    \hspace{1.4em}長方形$i$の高さの上界
    \end{tabular} \\ \hline
    
    $W$ &
    \begin{tabular}{l}
    \hspace{1.4em}母材の横幅
    \end{tabular} \\ \hline
    $H$ &
    \begin{tabular}{l}
    \hspace{1.4em}母材の高さ
    \end{tabular} \\ \hline
    $S_i$ &
    \begin{tabular}{l}
    \hspace{1.4em}長方形$i$の面積
    \end{tabular} \\ \hline
    $I$ &
    \begin{tabular}{l}
    \hspace{1.4em}詰め込む長方形の集合
    \end{tabular} \\ \hline
    
    \end{tabular}
    \end{center}
    \end{table}

\section{長方形詰込み問題}
長方形詰込み問題とは,母材の幅$W$と高さ$H$, 長方形集合$I$に対し, 幅$w_i$, 高さ$h_i$が入力として与えられ, 母材からはみ出ることなくまた, どの二つの長方形も互いに重なることがないように詰め込む問題である.\\
制約を定式化すると, 次のように書くことができる. \\
制約: 長方形$i \in I$は母材上に配置される. \\
\begin{eqnarray}
    0 \leq x_i \leq W-w_i (i \in I)\\
    0 \leq y_i \leq H-h_i (i \in I)
\end{eqnarray}
制約: 長方形$i,j \in I$は互いに重ならない. \\
この制約条件は, 次の4つの不等式のうち一つ以上が成立しなければいけない.  
\begin{eqnarray}
    x_i + w_i \leq x_j \\
    x_j + w_j \leq x_i \\
    y_i + h_i \leq y_j \\
    y_j + h_j \leq y_i
\end{eqnarray}

\section{soft-rectangleパッキング}
長方形詰込み問題の一種で,詰め込む長方形として幅と高さを変数としたsoft-rectangle (可変形状長方形)を扱う\cite{soft-rectangle}.\\
通常,面積に関する次のような等式制約または不等式制約が存在する.\\
\begin{eqnarray}
    w_i * h_i = S_i \\
    w_i * h_i \geq S_i \\
\end{eqnarray}
また,幅と高さの上界,下界に関する以下の制約も存在する.
\begin{eqnarray}
    w_i^l \leq w_i \leq w_i^u \\
    h_i^l \leq h_i \leq h_i^u
\end{eqnarray}

\section{Sequence-pair}
Sequence-pairとは,二つの順列による長方形配置表現である\cite{seq-pair}. 
長方形同士の相対位置関係を長方形名の順列対により表すことができる. 
Sequence-pairはどのような配置表現でも一意の順列対に表現でき,逆に任意の順列対から一意に配置表現を求めることができる. 


\section{no-fit polygon}
no-fit-polygon (NFP)とは, 平面上で多角形の重なりを判定する方法である\cite{nfp}\cite{nfp2}.
多角形$P$と$Q$が与えられ,$Q$の配置が固定されているとする.
この時,$Q$と重なりを持つ領域をNFP($P,Q$)と書き,図\ref{figure31}に示す.
図\ref{figure31}の太線がNFPの境界を表し,このNFP内に$P$を配置すると$Q$と重なることを意味する.
母材内の既配置の長方形全てに対し,長方形$i$のNFPを作成すると, $i$の配置可能領域はどのNFPの内部にも含まれない領域であるといえる.

\begin{figure}
    \label{figure31}
    % \includegraphics{}
    \caption{NFP($P,Q$)}
\end{figure}