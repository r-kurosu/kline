\chapter{定式化}\label{formulation}

この章では本研究における用語や制約,目的関数について説明する.
本研究では2段階に分けた構築法を提案するため,それぞれ分けて説明する.
それぞれの段階の説明は第4章に記す.

\section{記号の定義}

% \begin{table}[htb]
%     \caption{変数の定義}
%     \begin{center}
%     \label{table31}
%     \begin{tabular}{cp{35em}} \hline
%     変数 & \hspace{2.0em}変数の説明 \\ \hline
    
%     $w_i$ &
%     \begin{tabular}{l}
%     \hspace{1.4em}長方形iの横幅
%     \end{tabular} \\ \hline

%     $h_i$ &
%     \begin{tabular}{l}
%     \hspace{1.4em}長方形iの高さ
%     \end{tabular} \\ \hline

%     $x_i$ &
%     \begin{tabular}{l}
%     \hspace{1.4em}長方形iのx座標
%     \end{tabular} \\ \hline

%     $y_i$ &
%     \begin{tabular}{l}
%     \hspace{1.4em}長方形iのy座標
%     \end{tabular} \\ \hline

%     $L$ &
%     \begin{tabular}{l}
%     \hspace{1.4em}積み地の集合
%     \end{tabular} \\ \hline
    
%     $D$ &
%     \begin{tabular}{l}
%     \hspace{1.4em}揚げ地の集合
%     \end{tabular} \\ \hline

%     $S_i$ &
%     \begin{tabular}{l}
%     \hspace{1.4em}長方形iの面積
%     \end{tabular} \\ \hline

%     $w_i$ &
%     \begin{tabular}{l}
%     \hspace{1.4em}長方形iの横幅
%     \end{tabular} \\ \hline


%     \end{tabular}
%     \end{center}
%     \end{table}
    
%     \begin{table}[htb]
%     \caption{定数の定義}
%     \begin{center}
%     \label{table32}
%     \begin{tabular}{cp{35em}} \hline
%     定数 & \hspace{2.0em}定数の説明  \\ \hline
    
%     $W$ &
%     \begin{tabular}{l}
%     \hspace{1.4em}デッキの横幅
%     \end{tabular} \\ \hline
    
%     $H$ &
%     \begin{tabular}{l}
%     \hspace{1.4em}デッキの高さ
%     \end{tabular} \\ \hline
    
%     $L$ &
%     \begin{tabular}{l}
%     \hspace{1.4em}積み地の集合
%     \end{tabular} \\ \hline
    
%     $D$ &
%     \begin{tabular}{l}
%     \hspace{1.4em}揚げ地の集合
%     \end{tabular} \\ \hline

%     $O$ &
%     \begin{tabular}{l}
%     \hspace{1.4em}障害物の集合
%     \end{tabular} \\ \hline
    
%     $I_i$ &
%     \begin{tabular}{l}
%     \hspace{1.4em}グループiに属する車の集合
%     \end{tabular} \\ \hline

%     $n$ &
%     \begin{tabular}{l}
%     \hspace{1.4em}デッキ内のグループの数
%     \end{tabular} \\ \hline
    
%     $H$ &
%     \begin{tabular}{l}
%     \hspace{1.4em}デッキの高さ
%     \end{tabular} \\ \hline
    
%     \end{tabular}
%     \end{center}
%     \end{table}
    


\section{制約}
本研究で扱う独自の制約について説明する.
\subsection*{第一段階}
\textgt{(i)自動車移動経路に関する制約}\\
ランプから配置場所まで,自走で到達するためには,船の入口から配置場所まで移動する経路が必要である.
従って本研究では,必要な導線上に,既配置の車がないようにする.\\

\textgt{(ii)グループの大きさに関する制約}\\
各グループの
\subsection*{第二段階}
\textgt{(i)配置位置に関する制約}\\
各車の配置位置は,席割で決められたホールドに駐車しなければいけない.
実際には,ホールド単位ではなくセグメント単位で行う.\\

\textgt{(ii)駐車方法に関する制約}\\
駐車は原則的にバック駐車で行う.
配置場所付近に障害物等があると,物理的に駐車できない可能性がある.\\

\textgt{(iii)駐車間隔に関する制約}\\
駐車する車同士の間隔は,前後方向に10cm,左右方向に40cm空いていなければいけない.\\


\section{目的関数}
\subsection*{第一段階}
\subsection*{第二段階}