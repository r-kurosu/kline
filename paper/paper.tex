%#!ptex2pdf -l -u -ot '--synctex=1 --shell-escape' template
%\documentclass[platex]{jsarticle}
\documentclass[platex]{jreport}
% \usepackage[platex]{jreport}
\usepackage{booktabs}
\usepackage{amsmath}
\usepackage{amssymb} 
\usepackage{psfrag}
\usepackage[T1]{fontenc }
\usepackage{pifont}
\usepackage{algorithm}
\usepackage{algorithmic}
\usepackage{amsmath,cases}
\usepackage{here}
\usepackage[dvipdfmx]{graphicx}
\usepackage{url}      %URLの表記に使う\urlコマンドに必要.
\usepackage{enumerate}%enumerate環境で項目を[Step 1.]のような形式に変更するのに利用.
\usepackage{setspace}


\setlength{\topmargin}{-10mm}
\setlength{\textheight}{23cm}
\setlength{\oddsidemargin}{5mm}
\setlength{\evensidemargin}{5mm}
\setlength{\textwidth}{15cm}

\renewcommand{\tablename}{表}
\renewcommand{\figurename}{図}
%\newcommand{\bs}{\texttt{\symbol{'134}}}
%    \newcommand{\cmd}[1]{\texttt{\def\{{\char`\{}\def\}{\char`\}}\bs#1}}
\newtheorem{thm}{定理}[section]
\newtheorem{prf}{証明}[section]
\usepackage{latexsym}
\def\qed{\hfill $\Box$}

\usepackage[margin=3.25cm]{geometry}
\renewcommand{\bibname}{参考文献}

\begin{document}





%%%%%%%%%%%%%%%%%%%%%%%%%%%%%%%%%%%%%%%%%%%%%%%%%
%表紙
\begin{table}[b]
\begin{center}
{\huge 卒\hspace{0.1cm} 業\hspace{0.1cm} 論\hspace{0.1cm} 文}\\[2.5cm]
\begin{spacing}{1.5}
	{\huge 自動車運搬船における貨物積載プランニングの車両配置問題に対する構築型解法}\\[6cm]
\end{spacing}
{\huge 101810106\qquad 黒須諒}\\[1cm]
{\huge 名古屋大学情報学部}\\[0.5cm]
{\huge 自然情報学科数理情報系}\\[0.5cm]
{\huge 2022年2月}\\
\end{center}
\end{table} 
%%%%%%%%%%%%%%%%%%%%%%%%%%%%%%%%%%%%%%%%%%%%%%%%%


\thispagestyle{empty} 
\clearpage
\newpage
\pagenumbering{roman}
\setcounter{page}{1}


%%%%%%%%%%%%%%%%%%%%%%%%%%%%%%%%%%%%%%%%%%%%%%%%%
%摘要とabstract
\begin{center}
{\LARGE 自動車運搬船における貨物積載プランニングの車両配置計画に対する構築型解法}\\[0.5cm]
\end{center}
\hfill
{\large 101810106\qquad 黒須諒}\\[0.5cm]
\begin{center}
{\Large \bf 概 要}\\
\end{center}

複数の港で自動車を積み,複数の港で自動車を降ろす自動車運搬船について考える.
自動車運搬船に積む自動車の集合が与えられてから実際に自動車が船に積まれるまでに,席割とシミュレーションと呼ばれる二種類の作業が行われる.
席割では,船の階層内を一定間隔の大きさに区切ったホールドと呼ばれるスペースの各々に,与えられた積載自動車リスト喉の自動車を何台割り当てられるかを考える.
シミュレーションでは席割作業でホールド毎に割り当てられた自動車を,自動車の向きや空きスペース,作業効率などを考慮して車一台一台の配置場所を決定する.
本研究では,シミュレーションの自動化をするために数理最適化の技術によってコンピュータで短時間かつ効率の良いシミュレーション結果を出力することを目標とする.

シミュレーションは二次元パッキング問題として定式化することが出来るが,人が自動車を運転して所定の位置まで移動するということを考えなければいけない.
各デッキのランプと呼ばれる入口から配置場所までの移動経路確保と,駐車に要する局所的スペースの確保が課題となる.

本研究では二段階の構築法を提案する.
一段階目では,積み地,揚げ地が同じ車を1つのグループとし,グループの大まかな配置場所を決める.
グループに属する車の総面積を求め,面積を固定した,形状可変の長方形詰込み問題として定式化する.
各グループを積み港を優先して並べた配列と揚げ港を優先して並べた順列対を用意しシーケンスペア法を用いた制約を追加することで,ランプから配置場所までの移動経路を確保することができる.
この計算には,商用の整数計画ソルバー(Gurobi Optimizer)を使用し最適解を求める.

二段階目では,各グループ内の車を一段階目で決められた場所の中で一台ずつパッキングを行う.
解法として配置可能な場所の中で出来るだけ左下に詰込むbottom-left法を用いた.
この際,配置する予定の車に駐車に必要なスペースを加えレクトリニア図形とすることで駐車時の局所的なスペース確保を実現する.

以上の構築法により得られた配置図と実際に現場で使われている配置図とを比較し,解の評価を行う.
% 

%%%%%%%%%%%%%%%%%%%%%%%%%%%%%%%%%%%%%%%%%%%%%%%%%%%%%%%
%修士論文の場合は英語のアブストも必要なので以下を記述して下さい
%学位の場合はいらないので, 以下は消して下さい. 
\newpage
\begin{center}{\LARGE A column generation approach \\for the bus crew scheduling problem}\\[0.5cm]
\end{center}
\hfill {\large 101810106\qquad Ryo Kurosu}\\[0.5cm]
\begin{center}
{\large \bf Abstract}\\
\end{center}
Write the abstract here.
Unfortunately, if you want a master's degree, you must write the abstract of your master thesis
in English (in addition to the Japanese one) even if you write your thesis in Japanese.
However, if you are going to get a bachelor's degree (not a master's degree),
you don't need to do so.
Good luck!!




\thispagestyle{empty} 
\tableofcontents
\newpage
\setcounter{page}{1}
\pagestyle{plain}
\pagenumbering{arabic}



\chapter{はじめに}

複数の港で自動車を積み, 複数の港で自動車を降ろす自動車運搬船について考える. 
一般的に自動車運搬船は, 自動車を船の一定間隔で区切られたホールドと呼ばれるスペースにどの自動車を何台割り当てるかを決定する席割と呼ばれる工程を経て, その後席割で割り当てられた自動車に対して向きや場所を考慮して一台ずつ船内の領域に配置するシミュレーションと呼ばれる作業を行う. 
現状, 自動車を輸送する会社はこの作業を人手で行なっていることが多く, 席割作業に3時間, シミュレーション作業に4時間かかることから, これらの工程を自動化することが業務効率化に役立つと考えている. 

本研究では, 2つの工程のうちシミュレーションに対して, 数理最適化の技術によってコンピュータで短時間かつ効率の良いシミュレーション結果を出力することを目標とする. 

本研究で扱うシミュレーションの概要について述べる. 
席割の結果から, 各階層の各ホールドにどの種類の車を何台詰め込むかという情報が与えられる. 
この情報をもとにシミュレーションを行う. 

シミュレーションは, 車を長方形に近似することで長方形詰込み問題として定式化できる. 
本研究では, 二段階に分けた構築法を提案する. 
一段階目では, 各ホールド内に詰め込む車を, 積み地や揚げ地の情報をもとにグループ分けし, グループごとに大まかな配置場所を決める. 
二段階では, グループ内における車両一台一台の詳細な配置場所を決定する. 

本稿では, 第2章で問題に対する詳細な設定や, 本研究で扱う自動車運輸航海における専門用語について定義する. 
第3章では, 問題の定式化と定式化に必要な変数や定数の定義を行う. 
第4章では, 提案手法の詳細な説明を行う. 

\chapter{問題定義}\label{definition}

シミュレーション作業は各階層ごとに行う.
各階は4つのホールドと呼ばれる一定の広さで区切られたスペースが存在し,各ホールドに詰込む車の数,大きさ,種類,積み地,揚げ地は決まっている.
本稿では複数の種類のある階層の中で,単純な階層のデータを用いて行う.



\section{用語定義}
本研究で扱う船の航海等に関する専門用語の定義をする.

\begin{itemize}
    \item 席割 \\
    注文一つ一つを船のホールド割り当てる作業.

    \item シミュレーション \\
    席割で決まった自動車を一台ずつホールド内の領域に貼り付ける作業.

    \item  プランナー \\
    席割やシミュレーションを考える作業者.

    \item  オペレーター\\
    港で自動車をホールドまで運転して積み降ろしをする作業者.

    \item デッキ \\
    船の内部の階層.

    \item ホールド \\
    各デッキ内を一定間隔の領域で仕切られた空間.

    \item ランプ \\
    上下デッキに移動するために各デッキの特定ホールドについているスロープ.

    \item 注文 \\
    乗用車100台を港Aから港Bへ輸送, トラック30台を港Cから港Dへ輸送というような積載自動車の情報とそれらの積み地と揚げ地に関する情報. 

    \item 積み地 \\
    注文における自動車を積む港.

    \item 揚げ地 \\
    注文における自動車を降ろす港.

    \item  RT (revenue ton) \\
    基準となる車一台の面積に対する各注文の車の面積の割合. 

    \item ユニット数 \\
    各注文に含まれる自動車の台数.
\end{itemize}


\section{入力情報}
本研究で扱う2種類の情報について述べる.

\begin{itemize}
    \item 車情報 \\
    車の数,大きさ,積み地,揚げ地,指定のホールドを受け取る.
    \item 船体情報 \\
    船体の大きさ,各階層のランプの位置,船内の障害物の位置や大きさは実際のデータを元に手入力で行った.
\end{itemize}
入力情報のうち車体情報の例を表2.1に示す.

\section{出力}
図2.2のような配置図を出力とする.


\chapter{定式化}\label{formulation}

この章では本研究における用語や制約,目的関数について説明する.
本研究では2段階に分けた構築法を提案するため,それぞれ分けて説明する.
それぞれの段階の説明は第4章に記す.

\section{記号の定義}

% \begin{table}[htb]
%     \caption{変数の定義}
%     \begin{center}
%     \label{table31}
%     \begin{tabular}{cp{35em}} \hline
%     変数 & \hspace{2.0em}変数の説明 \\ \hline
    
%     $w_i$ &
%     \begin{tabular}{l}
%     \hspace{1.4em}長方形iの横幅
%     \end{tabular} \\ \hline

%     $h_i$ &
%     \begin{tabular}{l}
%     \hspace{1.4em}長方形iの高さ
%     \end{tabular} \\ \hline

%     $x_i$ &
%     \begin{tabular}{l}
%     \hspace{1.4em}長方形iのx座標
%     \end{tabular} \\ \hline

%     $y_i$ &
%     \begin{tabular}{l}
%     \hspace{1.4em}長方形iのy座標
%     \end{tabular} \\ \hline

%     $L$ &
%     \begin{tabular}{l}
%     \hspace{1.4em}積み地の集合
%     \end{tabular} \\ \hline
    
%     $D$ &
%     \begin{tabular}{l}
%     \hspace{1.4em}揚げ地の集合
%     \end{tabular} \\ \hline

%     $S_i$ &
%     \begin{tabular}{l}
%     \hspace{1.4em}長方形iの面積
%     \end{tabular} \\ \hline

%     $w_i$ &
%     \begin{tabular}{l}
%     \hspace{1.4em}長方形iの横幅
%     \end{tabular} \\ \hline


%     \end{tabular}
%     \end{center}
%     \end{table}
    
%     \begin{table}[htb]
%     \caption{定数の定義}
%     \begin{center}
%     \label{table32}
%     \begin{tabular}{cp{35em}} \hline
%     定数 & \hspace{2.0em}定数の説明  \\ \hline
    
%     $W$ &
%     \begin{tabular}{l}
%     \hspace{1.4em}デッキの横幅
%     \end{tabular} \\ \hline
    
%     $H$ &
%     \begin{tabular}{l}
%     \hspace{1.4em}デッキの高さ
%     \end{tabular} \\ \hline
    
%     $L$ &
%     \begin{tabular}{l}
%     \hspace{1.4em}積み地の集合
%     \end{tabular} \\ \hline
    
%     $D$ &
%     \begin{tabular}{l}
%     \hspace{1.4em}揚げ地の集合
%     \end{tabular} \\ \hline

%     $O$ &
%     \begin{tabular}{l}
%     \hspace{1.4em}障害物の集合
%     \end{tabular} \\ \hline
    
%     $I_i$ &
%     \begin{tabular}{l}
%     \hspace{1.4em}グループiに属する車の集合
%     \end{tabular} \\ \hline

%     $n$ &
%     \begin{tabular}{l}
%     \hspace{1.4em}デッキ内のグループの数
%     \end{tabular} \\ \hline
    
%     $H$ &
%     \begin{tabular}{l}
%     \hspace{1.4em}デッキの高さ
%     \end{tabular} \\ \hline
    
%     \end{tabular}
%     \end{center}
%     \end{table}
    


\section{制約}
本研究で扱う独自の制約について説明する.
\subsection*{第一段階}
\textgt{(i)自動車移動経路に関する制約}\\
ランプから配置場所まで,自走で到達するためには,船の入口から配置場所まで移動する経路が必要である.
従って本研究では,必要な導線上に,既配置の車がないようにする.\\

\textgt{(ii)グループの大きさに関する制約}\\
各グループの
\subsection*{第二段階}
\textgt{(i)配置位置に関する制約}\\
各車の配置位置は,席割で決められたホールドに駐車しなければいけない.
実際には,ホールド単位ではなくセグメント単位で行う.\\

\textgt{(ii)駐車方法に関する制約}\\
駐車は原則的にバック駐車で行う.
配置場所付近に障害物等があると,物理的に駐車できない可能性がある.\\

\textgt{(iii)駐車間隔に関する制約}\\
駐車する車同士の間隔は,前後方向に10cm,左右方向に40cm空いていなければいけない.\\


\section{目的関数}
\subsection*{第一段階}
\subsection*{第二段階}
\chapter{提案手法}\label{method}
本章では, 第3章で定式化した二段階の問題についてそれぞれの問題に対する提案手法を説明する. 

\section{第一段階}
求解には整数計画ソルバー (Gurobi Optimizer ver. 9.5)を用いた. 
第二段階では, この時の出力結果を利用する.  

\section{第二段階}
第二段階では, 第3章で説明した制約を満たしながら, 2種類のアルゴリズムを用意し構築を行う. 


\subsection{bottom-left法}
bottom-left法 (BL法)とは長方形詰込み問題に対するアルゴリズムの一つで, 配置可能な領域の内, 最も下に, 同じ高さの候補があれば, 最も左に配置する手法である\cite{nfp2}. 
面積の大きい順にパッキングするといったように, 詰め込む順番を工夫することで有用な解が得られるとされている. \\

\subsection{next-fit法}
next-fit法 (NF法)とは, ビンパッキング問題に対する近似解法を長方形詰込み問題に拡張した解法である\cite{next-fit}. 
容器を水平な直線でレベルと呼ばれる領域に分割して長方形を詰め込むため, レベル法とも呼ばれる. 
左端に設置した長方形が各レベルの高さを決定しており, 長方形を左端から順い横一列に配置し, 現在のレベルの中に配置できない長方形が現れた場合, 新しいレベルを作成し, その長方形を左端に詰め込む. \\

本研究では, 障害物などにより配置できない場合があるため, 以下のルールを考える. 
\begin{enumerate}
    \item 母材内に配置できるが, 障害物などで配置できない場合, 水平方向右向きに1 cmずらした点を新たな候補点とする
    \item 長方形が母材からはみ出るような場合は新しいレベルを作成する
\end{enumerate}


どちらの手法においても, グループ内の全ての車が詰め終わるか, 詰め込み途中で, グループ内のどの車も詰め込むことができない場合, 計算を終了し, 次のグループ内の車の詰め込みを開始した. 
また, プランナーの要望を考慮し, 搬出時にランプに向かって前向きに発進できるようにそれぞれの手法を修正し,  現場での積み降ろし作業の効率化の為, できるだけハンドル向きが同じ車が隣になるように, 以下の優先順位をもとに詰め込みを行った. 
\begin{enumerate}
    \item 同じ車種が連続する
    \item ハンドル向きが同じ車が連続する
    \item 長さの大きい車
    \item 幅の大きい車
\end{enumerate}


\section{局所探索法}
\subsection*{局所探索法}
ある解$x$に修正を加えることを近傍操作という\cite{local-search}. 
近傍操作により得られる解の集合$N(x)$を近傍と呼ぶ. 
局所探索法は, 適当な初期解から始め, 現在の解$x$よりも良い解$x'$が近傍$N(x)$内に存在すれば, $x:=x'$と置き換える操作を可能な限り繰り返す方法である. \\

本研究では, BL法とNF法により得られたそれぞれの初期解に近傍操作を加え改善を行う. 
評価関数としては, 詰め込むことのできた車の台数を用いる. \\
近傍操作では, 第一段階で決めたグループの大きさを変更する. 
隣接する2つのグループ$i,j$を比較し, グループ$i$では全ての車が詰め終わり, グループ$j$では詰め込むことのできなかった車が存在するとする. 
このようなグループ$j$の候補が複数ある場合, 最も詰め込むことのできなかった台数が多いグループを選ぶ. 
このとき, グループ$i$の面積を少し小さくし, グループ$j$の面積を少し大きくする. 
ここで変化させる面積はグループ$j$で詰め込むことのできなかった車の総面積分とする. 
以上の近傍操作を加えたのち, 再度, 第二段階の手法を用いて車を詰め込み, 最終的に詰め込むことのできた車の台数を元の解と比較する. 
局所探索の終了条件としては, 以下の通りである. 
\begin{itemize}
    \item デッキ内の全ての車を積み終えた時
    \item 近傍操作するグループが見つからなかった時
    \item 近傍操作によって解が改善されない時
\end{itemize}
% \include{chapter5}
\chapter{計算実験}\label{computational_result}
\section{計算環境}
実験に用いるプログラムはPythonを用いて実装し, 計算機はMacBookAir(CPU: Apple M1 chip, メモリ: 8GB)を用いて行った. 

\section{第一段階の求解結果}
整数計画ソルバー (Gurobi Optimizer ver. 9.5)を用いて計算した結果, 全ての問題例で最適解が得られた. 
出力の例を以下の図に示す. 
長方形内の番号はグループの番号を表す. 

\section{第二段階の求解結果}
出力される配置図と詰め込むことのできなかった車の台数の例をそれぞれ以下の図に示す. 


\chapter{まとめと今後の研究計画}\label{conclution}
自動車運搬船への貨物積み付け計画における, 車両配置計画に対して2段階の構築法を提案した. 
一段階目では, sequence-pairを用いた制約を追加し, soft-rectangleパッキング問題として定式化することで, 搬入搬出時の経路を確保した.  
ニ段階目では, レクトリニア図形を用いたパッキング問題として定式化し, 駐車時の局所的なスペースを確保した. 
第一段階では整数計画ソルバーを用い, 第二段階ではbottom-left法とnext-fit法を用いて解を構築した. 
さらに以上の構築法で得られた解を初期解とし, 近傍操作を行った. 
その結果, どちらの手法においても多くの問題例で解が改善され, より多くの車を詰め込むことができた. 

今後の課題として, 縦列駐車などの駐車方法を増やし, 最終的な配置図を実際に業務で使われているものに寄せていくことが挙げられる. 

\include{chapter8}




\addcontentsline{toc}{chapter}{参考文献}
\begin{thebibliography}{99}
	\bibitem{mitsui}
	「商船三井 プレスリリース」https://www.mol.co.jp/pr/2019/19072.html (2022年2月10日)
	\bibitem{takeda}
	竹田陽, "自動車運搬船における貨物積載プランニングの席割問題に対する局所探索," 修士論文, 名古屋大学情報学研究科, 2022.
	\bibitem{rect-pack}
	今堀慎治, 梅谷俊治, "切り出し・詰込み問題とその応用 -- (2) 長方形詰込み問題--," オペレーションズ・リサーチ, 50(2005), 335--340.
	\bibitem{soft-rectangle}
	T. Ibaraki and K. Nakamura, "Packing Problems with Soft Rectangles," LNCS 4030, (2006), 13--27.
	\bibitem{seq-pair}
	H. Murata, K. Fujiyoshi, S. Nakatake and Y. Kajitani, "VLSI Module Placement Based on Rectangle-Packing by the Sequence-Pair," 
	IEEE Transactions on Computer-aided Design of Integrated Circuits and Systems, 15(12), 1518--1524, December 1996.
	\bibitem{nfp}
	R.C. Art, "An approach to the two dimensional irregular cutting stock problem,"
	Ph.D. Thesis, Massachusetts Institute of Technology, 1966.
	\bibitem{nfp2}
	今堀慎治, 胡艶楠, 橋本英樹, 柳浦睦憲, "Pythonによる図形詰込みアルゴリズム入門," 
	オペレーションズ・リサーチ, 63(2008), 762--769.
	\bibitem{local-search}
	柳浦睦憲, "局所探索法—反復改善に基づく最適化の基本戦略," オペレーションズ・リサーチ, 52(2007), 538--542.
	\bibitem{next-fit}
	J.O. Berkey, P.Y. Wang, "Two-Dimensional Finite Bin-Packing Algorithms," 
	The Journal of the Operational Research Society, 38(1987), 423--429. 

\end{thebibliography}
\end{document}



% LocalWords:  ij Imahori Yagiura Ibaraki th Metaheuristics McGeoch Aarts lrr
% LocalWords:  Lenstra Chichester algorithmicx Fulkerson lrrcrr
%%presented by YUKI SAWAI 2015/12/22
%この卒論修論はhttp://www.co.cm.is.nagoya-u.ac.jp/~yagiura/writing/haifu_template/をもとに澤井佑樹が作っています
