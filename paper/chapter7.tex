\chapter{まとめ}\label{conclution}
まとめは著者から読者への締めくくりの言葉です.
すなわち,「この論文のポイントは結局何だったのか」を端的に読者に示す大切な部分です.
%すなわち,「この論文で言いたかったことは結局何だったのか」を端的に読者に示す大切な部分です.
必ず書きましょう\footnote{レター(ページ数の少ない速報的な論文
(e.g., Operations Research Letters, Information Processing Letters))などの
短いものではまとめの節を書かないように指示されることもあり,そのような場合を除く.}.
結局何をしてどうなったのかということを最後にもう一度手短にまとめて述べます.
この節で新しいこと(つまりこれまでの節で書いて来なかったこと)を書いてはいけません.

通常の論文等ではこのように得られた成果をまとめた結論を書いて終わるのですが,
研究の進捗状況を報告する普段の発表では,結論を書くことは難しいかもしれません.
また,既に一定の成果が得られている場合でも,卒論・修論の締切が差し迫った時期を除き,
卒論・修論に向けてさらに研究を進める予定であると思います.
そのような場合には,この節のタイトルを例えば「まとめと今後の研究計画」などとして,
まず現在までに得られている成果をまとめたのち,
今後の研究計画を簡潔に書いてください.