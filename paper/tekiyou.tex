\begin{center}
{\LARGE 自動車運搬船における貨物積載プランニングの\\車両配置計画に対する構築型解法}\\[0.5cm]
\end{center}
\hfill
{\large 101810106\qquad 黒須諒}\\[0.5cm]
\begin{center}
{\Large \bf 概 要}\\
\end{center}

複数の港で自動車を積み, 複数の港で自動車を降ろす自動車運搬船について考える. 
自動車運搬船に積む自動車の集合が与えられてから実際に自動車が船に積まれるまでに, 席割とシミュレーションと呼ばれる二種類の作業が行われる. 
席割では, 船の階層内を一定間隔の大きさに区切ったホールドと呼ばれるスペースの各々に, 与えられた積載自動車リストの自動車を何台割り当てられるかを考える. 
シミュレーションでは席割でホールド毎に割り当てられた自動車を, 自動車の向きや空きスペース, 作業効率などを考慮して車一台一台の配置場所を決定する. 
車両配置図を出力することで, 現場の作業員が配置図に従い, 効率よく駐車作業を行うことができる. 
本研究では, シミュレーションの自動化をするために数理最適化の技術によってコンピュータで短時間かつ効率の良いシミュレーション結果を出力することを目標とする. 

シミュレーションは長方形詰込み問題として定式化することが出来るが, 人が自動車を運転して所定の位置まで移動するということを考えなければいけない. 
そのため, 各デッキのランプと呼ばれる入口から配置場所までの移動経路確保と, 駐車に要する局所的スペースの確保が課題となる. 

本研究では二段階の構築法を提案する. 
一段階目では, 積み地, 揚げ地が同じ車を1つのグループとし, グループの大まかな配置場所を決定する. 
グループに属する車の総面積を求め, 面積を固定した, 形状可変長方形詰込み問題として定式化する. 
各グループを積み港を優先して並べた配列と揚げ港を優先して並べた順列対を用意しシーケンスペアを用いた制約を追加することで, ランプから配置場所までの移動経路を確保することができる. 
この計算には, 商用の整数計画ソルバー(Gurobi Optimizer)を使用し最適解を求める. 

二段階目では, 各グループ内の車を一段階目で決められた場所の中で一台ずつパッキングを行う. 
二段階目では, 第一段階で決められた領域で詳細な車両配置場所を決定する. 
配置予定の車を, 駐車に必要なスペースを加えたレクトリニア図形とすることで駐車時に必要な局所的スペースを確保する. 
構築型解法として配置可能な場所の中で出来るだけ左下に詰込むbottom-left法を用いた実装をした. 

以上の構築法により得られた配置図と実際に現場で使われている配置図とを比較し, 解の評価を行う. 
