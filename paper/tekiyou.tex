\begin{center}
\begin{spacing}{1.3}
    {\LARGE 自動車運搬船における貨物積載プランニングの\\車両配置計画に対する構築型解法}\\[0.5cm]
\end{spacing}
\end{center}
\hfill
{\large 101810106\qquad 黒須諒}\\[0.5cm]
\begin{center}
{\Large \bf 概 要}\\
\end{center}

複数の港で自動車を積み, 複数の港で自動車を降ろす自動車運搬船について考える. 
自動車運搬船に積む自動車の集合が与えられてから実際に自動車が船に積まれるまでに, 席割とシミュレーションと呼ばれる二種類の作業が行われる. 
席割作業では, 船の階層内を一定間隔の大きさに区切ったホールドと呼ばれるスペースの各々に, 与えられた積載自動車リストの自動車を何台割り当てられるかを考える. 
シミュレーション作業では席割作業でホールド毎に割り当てられた自動車を, 自動車の向きや空きスペース, 作業効率などを考慮して車一台一台の配置場所を決定する. 
車両配置図を出力することで, 現場の作業員が配置図に従い, 効率よく駐車作業を行うことができる. 
本研究では, シミュレーション作業の自動化を目的とし, 短時間かつ実用的なシミュレーション結果を出力することを目指す. 

シミュレーション作業では車の全幅と全長を各辺とした長方形に近似することで, 長方形詰込み問題として定式化することが出来るが, 貨物である自動車は配置位置まで自走するということを考えなければいけない. 
そのため, 積み降ろしのタイミングでは配置場所までの搬入・搬出経路の確保と, 駐車時に要する局所的スペースの確保が課題となる. 

本研究では二段階の構築法を提案する. 
一段階目では, 積み地, 揚げ地が同じ車を1つのグループとし, デッキ内におけるグループの大まかな配置場所を決定する. 
グループに属する車の総面積を求め, 面積制約をもち, 長方形の高さと幅を変数とするsoft-rectangleパッキング問題として定式化する. 
各グループの積み地・揚げ地に関する順序対を用意しsequence-pairを用いた制約を追加することで, 搬入・搬出時の移動経路を確保した解が得られる. 
この計算には, 商用の整数計画ソルバー(Gurobi Optimizer ver. 9.5)を使用した.  

二段階目では, 第一段階で決められた領域内で車両一台ずつの詳細な配置場所を決定する. 
配置予定の車を, 駐車に必要なスペースを加えたレクトリニア図形に近似し, レクトリニア図形の詰込み問題として定式化する. 
車の向きを考慮したbottom-left法を用いて実装した. 

以上の二段階による構築法により得られた車両配置図と実際に現場で使われている車両配置図とを比較し, 解の評価を行った. 
その結果, 多くの問題例で, 搬入搬出時の経路確保と, 駐車時の局所的スペース確保ができていることを確認した. 

