\begin{center}
\begin{spacing}{1.3}
    {\LARGE 自動車運搬船における貨物積載プランニングの\\車両配置計画に対する構築型解法}\\[0.5cm]
\end{spacing}
\end{center}
\hfill
{\large 101810106\qquad 黒須諒}\\[0.5cm]
\begin{center}
{\Large \bf 概 要}\\
\end{center}

複数の港で自動車を積み, 複数の港で自動車を降ろす自動車運搬船について考える. 
運搬船に積む自動車の集合が与えられてから実際に自動車が船に積まれるまでに, 席割とシミュレーションと呼ばれる二種類の作業が行われる. 
席割作業では, 船の階層内を一定間隔の大きさに区切ったホールドと呼ばれるスペースの各々に, 与えられた積載自動車リストの自動車を何台割り当てられるかを考える. 
シミュレーション作業では席割作業でホールド毎に割り当てられた自動車を, 自動車の向きや空きスペース, 作業効率などを考慮して車一台一台の配置場所を決定する. 
車両配置図を出力することで, 現場の作業員が配置図に従い, 効率よく駐車作業を行うことができる. 
本研究では, シミュレーション作業の自動化を目的とし, 短時間で実用的なシミュレーション結果を出力することを目指す. 

シミュレーション作業は, 車の幅と長さを各辺とした長方形に近似することで, 長方形詰込み問題として定式化することが出来るが, 貨物である自動車は配置位置まで自走するということを考えなければならない. 
そのため, 積み降ろしのタイミングでは配置場所までの搬入搬出経路の確保と, 駐車時に要する局所的スペースの確保が課題となる. 

本研究では二段階の構築法と局所探索を提案する. 
一段階目では, 積み地, 揚げ地が同じ車を1つのグループとし, 大まかな配置場所を決定する. 
グループに属する車の総面積を求め, 面積制約をもち長方形の高さと幅を変数とする可変形状長方形パッキング問題として定式化する. 
各グループの積み地, 揚げ地に関する順序対を用いて, 先に積むグループや後から降ろすグループは入口から見て奥に配置するというグループ間の相対位置関係に関する制約を追加する. 
あるグループよりも先に積み, 先に降ろすというグループに対しては, 左右の位置関係に関する制約を追加する. 
以上により, 搬入搬出時の移動経路を確保した解が得られる. 
この計算には, 商用の整数計画ソルバー(Gurobi Optimizer ver. 9.5)を使用した.  

二段階目では, 第一段階で決められた領域内で車両一台ずつの詳細な配置場所を決定する. 
配置予定の車を, 駐車に必要なスペースを加えたレクトリニア図形に近似し, レクトリニア図形の詰込み問題として定式化する. 
この問題を解く手法として, 車の向きを考慮し, bottom-left法とnext-fit法による二通りの構築法を提案する. 
この時, あるグループでは与えられた全ての車を詰め込むことができ, あるグループでは詰め込めない車が存在したとする. 
このようなグループの第一段階で作った境界を変更し, 再度第二段階の詰め込み, 解を改善するという方法を提案する. 

以上の提案手法により得られた車両配置図と実際に現場で使われている車両配置図とを比較し, 解の評価を行った. 
その結果, 多くの問題例において, 搬入搬出時の経路確保と駐車時の局所的スペースの確保ができていることを確認した. 
2つの構築法を比べた結果, bottom-left法では多くの車を詰め込むことができた一方, 上下左右の車が揃っておらず実務で用いる配置図とは乖離していた. 
next-fit法では, 上下左右の車を揃えることができ, 実際に使われている配置図に近づけることができた一方, 詰め込むことのできる車の台数は少なくなった.  
また, 提案手法において局所探索適用前の解よりも, 局所探索適用後の解の方がより多くの車を詰め込めることを確認した. 
