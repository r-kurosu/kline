\chapter{問題定義}\label{definition}
問題の定義を書くときには,何が与えられて(i.e., 入力)何を決めることが求められているのか
(i.e., 決定変数),および解の良さを判断する基準(i.e., 目的関数)を明確にしてください.
整数計画問題として自然な定式化が可能な場合はそのような定式化を与えるとよいと思います.
フォーマットの一例を挙げておきます:
\begin{align*}
 &\textrm{maximize}   && \sum_{j=1}^n p_j x_j \\
 &\textrm{subject to} && \sum_{j=1}^n w_{ij} x_j \le c_i, & \forall i \in I \\
 &                    && x_j \in \{0, 1\},                & \forall j \in J.
\end{align*}

関連文献等を調査して,問題に対して分かっていることも書きましょう.
研究内容によって書くべきことは変わりますが,たとえば
NP困難性などの計算の複雑さ,
近似精度保証の上界と下界,
多項式時間で解ける特殊ケースとそのような場合の計算量
などの理論的に解明されている性質や,
どのような問題例が代表的なベンチマーク問題例として利用されていて,
どの程度の規模までがどの程度の時間で厳密に解かれているか,
メタ戦略のような解法の比較の対象としてはどの程度の規模が対象になっているかなど.