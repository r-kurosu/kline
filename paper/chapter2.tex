\chapter{問題定義}\label{definition}

シミュレーション作業は各階層ごとに行う.
各階は4つのホールドと呼ばれる一定の広さで区切られたスペースが存在し,各ホールドに詰込む車の数,大きさ,種類,積み地,揚げ地は決まっている.
本稿では複数の種類のある階層の中で,単純な階層のデータを用いて行う.



\section{用語定義}
本研究で扱う船の航海等に関する専門用語の定義をする.

\begin{itemize}
    \item 席割 \\
    注文一つ一つを船のホールド割り当てる作業.

    \item シミュレーション \\
    席割で決まった自動車を一台ずつホールド内の領域に貼り付ける作業.

    \item  プランナー \\
    席割やシミュレーションを考える作業者.

    \item  オペレーター\\
    港で自動車をホールドまで運転して積み降ろしをする作業者.

    \item デッキ \\
    船の内部の階層.

    \item ホールド \\
    各デッキ内を一定間隔の領域で仕切られた空間.

    \item ランプ \\
    上下デッキに移動するために各デッキの特定ホールドについているスロープ.

    \item 注文 \\
    乗用車100台を港Aから港Bへ輸送, トラック30台を港Cから港Dへ輸送というような積載自動車の情報とそれらの積み地と揚げ地に関する情報. 

    \item 積み地 \\
    注文における自動車を積む港.

    \item 揚げ地 \\
    注文における自動車を降ろす港.

    \item  RT (revenue ton) \\
    基準となる車一台の面積に対する各注文の車の面積の割合. 

    \item ユニット数 \\
    各注文に含まれる自動車の台数.
\end{itemize}


\section{入力情報}
本研究で扱う2種類の情報について述べる.

\begin{itemize}
    \item 車情報 \\
    車の数,大きさ,積み地,揚げ地,指定のホールドを受け取る.
    \item 船体情報 \\
    船体の大きさ,各階層のランプの位置,船内の障害物の位置や大きさは実際のデータを元に手入力で行った.
\end{itemize}
入力情報のうち車体情報の例を表2.1に示す.

\section{出力}
図2.2のような配置図を出力とする.

